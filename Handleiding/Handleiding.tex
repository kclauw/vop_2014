%%========================================================================
%% LaTeX scriptiesjabloon
%%========================================================================
%%========================================================================
%% Preamble
%%========================================================================

\documentclass[pdftex,a4paper,12pt,twoside]{report}

\usepackage{color}					 % kleur voor syntax highlighting
\usepackage{caption}
\usepackage{subcaption}
\usepackage[utf8]{inputenc}  % Accenten gebruiken in tekst (vb. � ipv \'e)
\usepackage{amsfonts}        % AMS math packages: extra wiskundige
\usepackage{amsmath}         %   symbolen (o.a. getallen-
\usepackage{amssymb}         %   verzamelingen N, R, Z, Q, etc.)
\usepackage[UKenglish]{babel}    % Taalinstellingen: woordsplitsingen,
                             %  commando's voor speciale karakters
                             %  ("dutch" voor NL)
                                                         %  ("UKenglish" voor brits engels)
\usepackage{eurosym}         % Euro-symbool �
\usepackage{graphicx}        % Invoegen van tekeningen
\usepackage[pdftex,bookmarks=true]{hyperref}
                             % PDF krijgt klikbare links & verwijzingen,
                             %  inhoudstafel
\usepackage{listings}        % Broncode mooi opmaken
\usepackage{multirow}        % Tekst over verschillende cellen in tabellen
\usepackage{rotating}        % Tabellen en figuren roteren
\usepackage{natbib}          % Betere bibliografiestijlen
\usepackage{fancyhdr}        % Pagina-opmaak met hoofd- en voettekst
\usepackage{footnote}
\usepackage{parskip}
\usepackage{longtable}			 % For tables longer than one page


% paragrafen zonder indentatie, en op andere lijn
\setlength{\parindent}{0pt }
\setlength{\parskip}{12pt plus 1pt minus 1pt}


\definecolor{dkgreen}{rgb}{0,0.6,0}
\definecolor{gray}{rgb}{0.5,0.5,0.5}
\definecolor{mauve}{rgb}{0.58,0,0.82}
\definecolor{darkgray}{rgb}{0.662745,0.662745,0.662745}
\definecolor{black}{rgb}{0,0,0}
\definecolor{lightgray}{rgb}{.9,.9,.9}
\definecolor{darkgray}{rgb}{.4,.4,.4}
\definecolor{purple}{rgb}{0.65, 0.12, 0.82}
\definecolor{darkblue}{rgb}{0.0,0.0,0.6}
\definecolor{cyan}{rgb}{0.0,0.6,0.6}


%%---------- Layout ------------------------------------------------------
\newcommand{\includecode}[2][c]{\lstinputlisting[caption=#2, escapechar=]{#2}}

% hoofdingen, enz.
\pagestyle{fancy}

% lijn, wordt gebruikt in titelpagina
\newcommand{\HRule}{\rule{\linewidth}{0.5mm}}

% Leeg blad
\newcommand{\emptypage}
{
	\newpage
	\thispagestyle{empty}
	\mbox{}
	\newpage
}
 
% Gebruik een schreefloos lettertype ipv het "oubollig" uitziende
% Computer Modern
\renewcommand{\familydefault}{\sfdefault}     

% Commando voor invoegen Java-broncodebestanden (dank aan Niels Corneille)
% Gebruik: \codefragment{source/MijnKlasse.java}{Uitleg bij de code}
\newcommand{\codefragmentjava}[2]
{ \lstset{%
  language=java,
  breaklines=true,
  float=th,
  caption={#2},
  basicstyle=\scriptsize,
  frame=single
}
\lstinputlisting{#1}}


\lstset{frame=tb,
  language=Java,
  aboveskip=3mm,
  belowskip=3mm,
  showstringspaces=false,
  columns=flexible,
  basicstyle={\small\ttfamily},
  numbers=left,
  numberstyle=\footnotesize,
  keywordstyle=\color{blue},
  commentstyle=\color{dkgreen},
  stringstyle=\color{mauve},
  breaklines=true,
  breakatwhitespace=true
  tabsize=3,
	captionpos=b,
	extendedchars=true
}

\lstdefinelanguage{JavaScript}
{
  keywords={typeof, new, true, false, catch, function, return, null, catch, switch, var, if, in, while, do, else, case, break},
  keywordstyle=\color{blue}\bfseries,
  ndkeywords={class, export, boolean, throw, implements, import, this},
  ndkeywordstyle=\color{darkgray}\bfseries,
  identifierstyle=\color{black},
  sensitive=false,
  comment=[l]{//},
  morecomment=[s]{/*}{*/},
  commentstyle=\color{purple}\ttfamily,
  stringstyle=\color{red}\ttfamily,
  morestring=[b]',
  morestring=[b]"
}

\lstdefinelanguage{XML}
{
  basicstyle=\ttfamily\color{darkblue}\bfseries,
  morestring=[b]",
  morestring=[s]{>}{<},
  morecomment=[s]{<?}{?>},
  stringstyle=\color{black},
  identifierstyle=\color{darkblue},
  keywordstyle=\color{cyan},
  morekeywords={xmlns,version,type}% list your attributes here
}


\lstdefinelanguage{CSS}
{
	alsodigit={-},
	ndkeywords={@import, @media, @page, @font-face, @charset, @namespace, @viewport, @-ms-viewport, @-o-viewport, @-moz-viewport, @-webkit-viewport, th},
  ndkeywordstyle=\color{darkgray}\bfseries,
  morekeywords={accelerator,azimuth,background,background-attachment,
    background-color,background-image,background-position,
    background-position-x,background-position-y,background-repeat,
    behavior,border,border-bottom,border-bottom-color,
    border-bottom-style,border-bottom-width,border-collapse,
    border-color,border-left,border-left-color,border-left-style,
    border-left-width,border-right,border-right-color,
    border-right-style,border-right-width,border-spacing,
    border-style,border-top,border-top-color,border-top-style,
    border-top-width,border-width,bottom,caption-side,clear,
    clip,color,content,counter-increment,counter-reset,cue,
    cue-after,cue-before,cursor,direction,display,elevation,
    empty-cells,filter,float,font,font-family,font-size,
    font-size-adjust,font-stretch,font-style,font-variant,
    font-weight,height,ime-mode,include-source,
    layer-background-color,layer-background-image,layout-flow,
    layout-grid,layout-grid-char,layout-grid-char-spacing,
    layout-grid-line,layout-grid-mode,layout-grid-type,left,
    letter-spacing,line-break,line-height,list-style,
    list-style-image,list-style-position,list-style-type,margin,
    margin-bottom,margin-left,margin-right,margin-top,
    marker-offset,marks,max-height,max-width,min-height,
    min-width,-moz-binding,-moz-border-radius,
    -moz-border-radius-topleft,-moz-border-radius-topright,
    -moz-border-radius-bottomright,-moz-border-radius-bottomleft,
    -moz-border-top-colors,-moz-border-right-colors,
    -moz-border-bottom-colors,-moz-border-left-colors,-moz-opacity,
    -moz-outline,-moz-outline-color,-moz-outline-style,
    -moz-outline-width,-moz-user-focus,-moz-user-input,
    -moz-user-modify,-moz-user-select,orphans,outline,
    outline-color,outline-style,outline-width,overflow,
    overflow-X,overflow-Y,padding,padding-bottom,padding-left,
    padding-right,padding-top,page,page-break-after,
    page-break-before,page-break-inside,pause,pause-after,
    pause-before,pitch,pitch-range,play-during,position,quotes,
    -replace,richness,right,ruby-align,ruby-overhang,
    ruby-position,-set-link-source,size,speak,speak-header,
    speak-numeral,speak-punctuation,speech-rate,stress,
    scrollbar-arrow-color,scrollbar-base-color,
    scrollbar-dark-shadow-color,scrollbar-face-color,
    scrollbar-highlight-color,scrollbar-shadow-color,
    scrollbar-3d-light-color,scrollbar-track-color,table-layout,
    text-align,text-align-last,text-decoration,text-indent,
    text-justify,text-overflow,text-shadow,text-transform,
    text-autospace,text-kashida-space,text-underline-position,top,
    unicode-bidi,-use-link-source,vertical-align,visibility,
    voice-family,volume,white-space,widows,width,word-break,
    word-spacing,word-wrap,writing-mode,z-index,zoom},
  morestring=[s]{:}{;},
	moredelim=[is][\color{black}\bfseries]{@*}{*@},
	moredelim=[is][\color{mauve}\bfseries]{@.}{.@},
	moredelim=[is][\color{blue}\bfseries]{@~}{~@},
	moredelim=[is][\color{red}\bfseries]{@�}{�@},
  sensitive,
  morecomment=[s]{/*}{*/}
}

%%---------- Documenteigenschappen ---------------------------------------
%% Vul dit aan met je eigen info:

% Je eigen naam
\newcommand{\studenta}{Kenzo Clauw}
\newcommand{\studentb}{Axl Fran\c{c}ois}
\newcommand{\studentc}{Lowie Huyghe}
\newcommand{\studentd}{Sander Trypsteen}
\newcommand{\studente}{Jelle Verreth}

% De naam van je stage-/bachelorproefbegeleider
%\newcommand{\begeleider}{} 

% De naam (én firma/organisatie) van je mentor/promotor
% Laat in commentaar indien niet van toepassing
%\newcommand{\mentor}{Jan Janssen, ACME Inc.}

% De titel van je scriptie/stageverslag
\newcommand{\titel}{VOP Project Dossier}

% Ondertitel
\newcommand{\ondertitel}{ Stambomen}

% Datum van indienen
\newcommand{\datum}{XX XXXX 2014}

% Academiejaar
\newcommand{\academiejaar}{2013-2014}

%%========================================================================
%% Inhoud document
%%========================================================================

\begin{document}

%%---------- Front matter ------------------------------------------------
%% Het voorblad - Hier moet je in principe niets wijzigen.

\begin{titlepage}
\begin{center}
\includegraphics[width=4cm]{images/logo.png}\\[.5cm]
Master of Science Industrial Science : Informatics\\
Academic year \academiejaar

\vfill

\HRule \\[0.4cm]
{ \huge \bfseries \titel}\\[0.4cm]
\HRule \\[0.4cm]

{\Large \ondertitel}\\[0.4cm]

Submitted on \datum

\vfill

% Studenten en begeleiders
\begin{minipage}{0.49\textwidth}
\begin{flushleft}
\emph{Student\ifdefined\studentb s\fi :}\\
\studenta \\
\studentb \\
\studentc \\
\studentd \\
\studente
\par
\end{flushleft}
\end{minipage}
\begin{minipage}{0.49\textwidth}
\begin{flushright}
%\emph{Tutor:}\\ \begeleider\\
%\ifdefined\mentor \emph{Mentor:}\\ \mentor \fi
\end{flushright}
\end{minipage}

\end{center}

\end{titlepage}

% Schutblad

\emptypage

% Herhaling titelblad

\begin{titlepage}
\begin{center}
Master of Science Industrial Science : Informatics\\
Academic year \academiejaar

\vfill

\HRule \\[0.4cm]
{ \huge \bfseries \titel}\\[0.4cm]
\HRule \\[0.4cm]

{\Large \ondertitel}\\[0.4cm]

Submitted on \datum

\vfill

% Studenten en begeleiders
\begin{minipage}{0.49\textwidth}
\begin{flushleft}
\emph{Student\ifdefined\ \fi :}\\
\studenta \\
\studentb \\
\studentc \\
\studentd \\
\studente
%\ifdefined\studentb \studentb \fi\par
\end{flushleft}
\end{minipage}
\begin{minipage}{0.49\textwidth}
\begin{flushright}
%\emph{Tutor:}\\ \begeleider\\
%\ifdefined\mentor \emph{Mentor:}\\ \mentor \fi
\end{flushright}
\end{minipage}

\end{center}

\end{titlepage}

%% Inhoudstafel
\abstract


\tableofcontents
\chapter{Inleiding}\label{ch:preface}
Om het gebruik van de applicatie te vereenvoudigen kunt u deze handleiding raadplegen die alle mogelijke functionaliteiten uit onze applicatie bevat.
Bij de handleiding worden de verschillende stappen van de applicatie uitgelegd aan de hand van een voorbeeld gebruiker Kenzo en een Admin die uiteraard over administrator mogelijkheden beschikt.


\chapter{Toegangsbeheer}\label{ch:preface}
\subsection{Login}
Bij het opstarten van de applicatie krijgt u toegang tot het login scherm die er als volgt uitziet :

\includegraphics[width=12cm]{images/login.png}\\[.5cm]

\includegraphics[width=12cm]{images/add_tree.png}\\[.5cm]
Er zijn 2 mogelijkheden om als gebruiker in te loggen.

\begin{enumerate}
\item \label{it:first}Inloggen door met gebruikersnaam en passwoord.



\item \label{it:first}Inloggen via facebook

\includegraphics[width=12cm]{images/facebook.png}\\[.5cm]
\end{enumerate}

\subsection{Registratie}
Op het loginscherm kan je een gebruiker aanmaken door een naam en passwoord op te geven : 

\includegraphics[width=12cm]{images/register.png}\\[.5cm]
Het passwoord moet minimum 8 characters lang zijn.
Na het registreren wordt je terug doorverwezen naar het login scherm.


\chapter{Overzicht stambomen}\label{ch:preface}
Na het inloggen als een gewone gebruiker krijg je een overzicht van stambomen.

\includegraphics[width=12cm]{images/user_treeoverview.png}\\[.5cm]

Een stamboom toevoegen kan via het menu Tree.
\includegraphics[width=12cm]{images/tree_add.png}\\[.5cm]

Er verschijnt een panel je een naam moet opgeven en 1 van de volgende privacy instellingen selecteren :

\includegraphics[width=12cm]{images/add_tree.png}\\[.5cm]

\begin{enumerate}
\item \label{it:first}Private
\includegraphics[width=12cm]{images/facebook.png}\\[.5cm]
\end{enumerate}

Na het toevoegen van een tree krijg je terug een overzicht van de stambomen.

\includegraphics[width=12cm]{images/user_treeoverview_full.png}\\[.5cm]

\chapter{Settings}
\subsection{Taal}
Om de taal van de applicatie te wijzigen :
\includegraphics[width=12cm]{images/change_language.png}\\[.5cm]
\subsection{Gedcom}
Om een gedcom bestand te importeren :
\includegraphics[width=12cm]{images/import_gedcom.png}\\[.5cm]


\chapter{Administrator}


\chapter{Web-applicatie}
\section{Toegangsbeheer}
De manier waarop een gebruiker zich kan registeren of kan inloggen gebeurt op dezelfde manier als bij de desktop-applicatie.
\section{Bomen}
\subsection{Overzicht}
Net als in de desktop-applicatie krijg je na het inloggen een overzicht van uw persoonlijke bomen. Er kan worden gezocht op publieke bomen of bepaalde bomen van vrienden met de zoekfunctie rechtsboven.

\includegraphics[width=12cm]{images/web_treeoverview.png}\\[.5cm]
Elk van de bomen kan worden gedeeld op Facebook met de share-knop.

\subsection{Boom}
Door één van de bomen in het overzicht aan te klikken ga je naar de boom in detail. Alle personen worden volgens de relationele structuur weergegeven.

\includegraphics[width=12cm]{images/web_tree.png}\\[.5cm]
Door een persoon aan te klikken krijg je een detail-overzicht te zien van die persoon zijn gegevens. Indien je te maken hebt met grote bomen kan de persoon in kwestie worden ingesteld als referentie-persoon. Vervolgens zal de boom vanuit het perspectief van die persoon worden getekend.

\includegraphics[width=12cm]{images/web_treedetail.png}\\[.5cm]
Om terug een volledig overzicht te krijgen van alle personen in de boom klik je op het 'refresh'-icoon in de linkerbovenhoek.

In het overzicht van een boom kan er rechtsboven worden gekozen om de boom weer te geven in de teletijdsmachine. Dit geeft je een overzicht van de geboorteplaatsen van alle personen in de boom die op een bepaald moment leefden. Elk van de markers stelt een persoon voor en kan worden aangeklikt. Een venster wordt geopend met een overzicht van de gegevens van die persoon.

\includegraphics[width=12cm]{images/web_timemachine.png}\\[.5cm]
De huidige datum kan worden gewijzigd door op de datum te klikken of door gebruik te maken van de schuifbar. Je kan ook de evolutie van de boom bekijken met een animatie door op de 'play'-knop te drukken. Indien je wil versnellen of vertragen gebruik je de schuifbar rechtsboven.

\subsection{Vrienden}
Rechtsboven kan er in de menu gekozen worden voor vrienden. Hier krijg je een overzicht van alle gebruikers die bevriend zijn met jouw of die jou willen toevoegen. Indien jouw vrienden wijzigingen aanbrengen in hun stambomen krijg jij een overzicht te zien van de wijzigingen die zijn gebeurt.

\includegraphics[width=12cm]{images/web_friends.png}\\[.5cm]

\includegraphics[width=12cm]{images/web_addfriend.png}\\[.5cm]
Een vriend kan worden toegevoegd met de voorziene knop. Hier voeg je de specifieke naam in van de gebruiker die je als vriend wilt toevoegen of kies je een Facebook-vriend uit de lijst indien je met Facebook bent ingelogd en er iemand van jouw vriendenlijst in de applicatie is geregistreerd.

\subsection{Publieke gebruikers}
Hier krijg je een overzicht van alle publieke gebruikers in het systeem. Er kan worden gezocht op gebruikersnaam om een specifieke gebruiker terug te vinden.

\includegraphics[width=12cm]{images/web_publicuser.png}\\[.5cm]

\subsection{Instellingen}
Onder instellingen kunnen er enkele configuratie-wijzigingen worden gemaakt. Zo kan de je voor uw account instellen of je deze al dan niet publiek wil zetten. Je kan ook de taal instellen of het thema van de applicatie wijzigen.

\includegraphics[width=12cm]{images/web_theme.png}\\[.5cm]


%%%%%%%%%%%%%%%%%%%%%%%%%%%%%%%%%%%%%%%%%%%%%%%%%%%%%%%%%%%%%
%% APPENDICES
%%%%%%%%%%%%%%%%%%%%%%%%%%%%%%%%%%%%%%%%%%%%%%%%%%%%%%%%%%%%%

\appendix
%\input{FileName} %You need a file 'FileName.tex' for this.

\end{document}

